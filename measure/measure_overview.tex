\documentclass[article]{article}
\usepackage{mathtools}
\usepackage{amssymb}
\usepackage{amsthm}
\begin{document}
	
	\title{Measure: Outline and Bibliography}
	\author{Rich Cochrane}
	\maketitle
	
\section*{Remark}

This part of the project is the least developed by far and may end up being dropped entirely, severely curtailed and integrated into Continuity, or perhaps even expanded. This is the lowest-priority part of the project so a decision will be made on this later.


\section{Measure Theory}

\begin{itemize}
	\item{Adams and Guillemin, \textit{Measure Theory and Probability}}
	\item{Federer, \textit{Geometric Measure Theory}}
\end{itemize}

It's likely that we'll need the notion of integral from Smooth Manifolds, so that will be a prerequisite, but I'm not ruling out starting all over again with a fresh idea of measurable spaces. After all, here we'll be integrating a much wider range of objects than just differential forms.

In fact I'm tempted -- at least from a research perspective -- to try to tie this to our approach to topology by looking at measurable spaces as a different subcategory of the category of locales from the one we study in topology. This paper is rather terse but contains all the pointers: arxiv.org/pdf/2005.05284. The question then is how to describe the relationship between the two in a way that clarifies things rather than obfuscates them.

\section{Probability and Statistics}

I'm not sure I have much to say here, as such, since it's a fairly straightforward progression from probability theory to stats. It would be worth seeing whether the big ideas of Continuity can illuminate any of this. 

\begin{itemize}
	\item{Grimmett and Stirzaker, \textit{Probability and Random Processes}}
	\item{Borovkov, \textit{Probability Theory}}
	\item{Wasserman, \textit{All of Statistics}}
\end{itemize}

\section{Calculus of Variations}

I'm not familiar enough with this field to even sketch a course yet. Giaquinta and Hildebrandt align the first and second variations with necessary and sufficient conditions (for a function to have extrema, I think) -- this is obviously an appealing idea but it's probably not all that deep. Their books look like a good historical survey though.

This is perhaps the place to study PDEs -- I've tried to avoid an explicit course on differential equations as I expect them to naturally weave through the Smooth Manifolds, Lie Groups and Differential Topology courses. But I feel like we could give more dedicated time to them here.

\section{Nonsmooth Analysis}

\begin{itemize}
	\item{Buchanan, \textit{An Undergraduate Introduction to Financial Mathematics}}
	\item{Jorgensen and Treadway, \textit{Analysis and Probability: Wavelets, Signals, Fractals}}
	\item{Kigami, \textit{Analysis on Fractals}}
\end{itemize}


	
\end{document}