\documentclass[article]{article}
\usepackage{mathtools}
\usepackage{amssymb}
\usepackage{amsthm}
\begin{document}

\title{Structure: Outline and Bibliography}
\author{Rich Cochrane}
\maketitle

\section{\textit{Hors d'oeuvre}}

Eventually there should be a course that gathers up all the absolutely essential algebra for the early continua courses, allowing a student to bypass the courses below that go into more detail. My approach is to gather up the prerequisites as we go and make this one of the last courses I write. Some things I know will need to go in there:

\begin{itemize}
	\item{Categories}
	\item{Sets}
	\item{Groups}
	\item{Exact sequences and homology groups}
	\item{Rings and fields}
	\item{Modules and vector spaces}
\end{itemize}

In general the student probably needs, for each kind of object,

\begin{itemize}
	\item{A definition of the object and its morphisms}
	\item{A few canonical examples}
	\item{What the kernels and cokernels look like}
	\item{Examples of subobjects, products, quotients}
\end{itemize}

The idea is to trade off the fact that this will be a bit dry with keeping it short and to the point, probably with a view to it being used more as a quick reference than a stand-alone course of study.

It may be worth including a bit more of the language of categories since I'm not sure where else we would do this. This part could be written earlier on.

\section{Group Theory}

The introductory course is already written but will need to be converted to \LaTeX. It's currently very traditional and could be expanded with more interesting material but I think many of the topics I'm interested in will come up more naturally in other courses anyway.

Lots of follow-up courses are possible: I'd be especially interested in doing one on braids (`dances of particles') and one on Coxeter groups (`systems of mirrors'). 

I expect group cohomology and topological groups will be adequately covered in the Lie Algebras block.

\section{Linear Algebra}

The introductory course is already written but will need to be converted to \LaTeX. A follow-up course based around applications to data analysis is also already written and should be kept separate as that doesn't help with prerequisites for later courses.

It's possible we want an advanced course on infinite-dimensional spaces and $C^\star$ algebras. For that I like the look of Sakai, \textit{$C^\star$ Algebras and $W^\star$ Algebras}, because the approach there seems to be very algebraic. This ties into a possible approach to measure theory, and that might end up determining whether we actually want to go there or not. It's possible that this ends up being part of the Measure Theory block instead, we'll have to see later.

\section{Homological Algebra}

I'd like to start from the position that homological algebra is a toolkit for solving problems, which means we're not really studying things like exact sequences for their own sake but rather learning techniques that deploy them to solve other kinds of problems. It's therefore plausible that we might make this dependent on the homology part of Algebraic Topology, since that would mean the student came to it with lots of material for examples. Otherwise it's probably a bit too dry.

Another way to do this is to \emph{show} things like the long exact sequence in Algebraic Topology but everything about it has to be taken on trust. We can then dig into the background of how and why they work in this course. That would mean the student of Homological Algebra would be advised to have studied Algebraic Topology previously, but it wouldn't be strictly required.

Rotman is huge but it may be worth it to work through the whole thing; Stammbach offers a more direct route through the essentials. This area overlaps heavily with the topology section Munkres so it might not be a whole separate course but it would be worth seeing whether, by going deeper into the abstract structure, we can demonstrate the broad usefulness of these techniques outside topology.

Bjork is a bit of a weird inclusion here but I'd like to see whether there are applications of homological algebra there that are unusual and potentially make connections to other sections of the project.

Philosophically I like the characterisation that homology `measures the gap between ought and is'. But more deeply, if maths is taken to be maximally rational we could say homology measures the shortfall of our current conception of it: if we saw things correctly, we would no longer have a mismatch between what we expect to be true (`ought') and what is.

My guess is this is two courses, just an introduction covering the essentials (maybe culminatign in Ext and Tor with some applications) and a more advanced follow-up with interesting topics (e.g. spectral sequences).

\begin{itemize}
	\item[]{Stammbach, \textit{A Course in Homological Algebra} (first c170pp)}
	\item[]{Rotman, \textit{Introduction to Homological Algebra}}
	\item[]{Bjork, \textit{Rings of Differential Operators}}
\end{itemize}

\section{Galois Theory}

I think the theme here is that Galois theory provides a way to understand extension problems and obstructions to solving them, which gives it a homological flavour. I'm sure all these references won't be needed but I don't know this territory.

There is some rationale to these choices, although some of this may turn out to be misplaced. Roman is introductory. Berhuy makes the link with cohomology, which is one way to connect us to the main trunk of the project; another is the connection with differential algebra made by Pommaret. Szamuealy Ch 1 and 2 provide a connection with algebraic topology that should be more elementary than Berhuy.

I suspect Malle and Matzat offer one route towards `Galois theory in itself' rather than as a technique for solving problems in other fields. Borceax and Janelidze appear to give a higher-level view of `Galois theory as a whole', providing more opportunities for structural connections that span multiple topics.

My guess is this is one long interesting-for-its-own-sake course but it might end up being three: an introductory one, one ephasizing connections with algebraic topology, another emphasizing the more categorical Galois-theory-in-general perspective.

\begin{itemize}
	\item[]{Roman, \textit{Field Theory} (first c200 pages)}
	\item[]{Berhuy, \textit{An Introduction to Galois Cohomology and its Applications}}
	\item[]{Malle and Matzat, \textit{Inverse Galois Theory}}
	\item[]{Pommaret, \textit{Differential Galois Theory}}
	\item[]{Szamuely, \textit{Galois Groups and Fundamental Groups}}
	\item[]{Borceux and Janelidze, \textit{Galois Theories}}
\end{itemize}

\section{Lie Algebras and Representation Theory}

This will be an \textit{algebraic} course rather than a geometric look at Lie groups but there is obvious overlap and it's not yet clear to me how much of this material belongs with differential geometry. Pittner is included specifically for advanced stuff on Lie groups and the connection with non-commutative objects that comes up in various other places.

I'm keeping Alexandrino and Bettiol, \textit{Lie Groups and Geometric Aspects of Isometric Actions} in reserve for two reasons. First, it's intended for undergraduates so it likely to have some pedagogic features. Second, it may have some nice geometric examples and applications that aren't covered elsewhere. However, they make Riemannian geometry a prerequisite and I'd prefer to make our Lie Algebras material accessible straight after Smooth Manifolds so their approach might end up being of limited use to us. It could even be that we steal some ideas from them for the Riemannian Geometry courses instead.

Probably this will be two courses: one very algebraic one with representations front and centre, the other more geometric. It's possible there's another course here on other kinds of associative algebra, probably with a prerequisite of Homological Algebra and drawing on some topics from Pierce, \textit{Associative Algebras}.

\begin{itemize}
	\item[]{Fulton and Harris, \textit{Representation Theory: A First Course}}
	\item[]{Hochschild, \textit{Basic Theory of Algebraic Groups and Lie Algebras}}
	\item[]{Brown, \textit{Cohomology of Groups}}
	\item[]{Pittner, \textit{Algebraic Foundations of Non-Commutative Differential Geometry and Quantum Groups}}
\end{itemize}

\section{Hopf Algebras}

Dascalescu goes slowly and has some good examples that are relevant to us. This is probably material to scatter throughout several parts of the project so it might not be a separate course. On the other hand since it comes up in several places it might be best to have a single source to learn it from.

There are nice diagrams to help with understanding of these -- see the note `Cohomology of the Adjoint Hopf Algebras' for an explanation with examples. It may be worth trying to include the same idea for simpler structures if it's possible to make them useful.

The following is mentioned in the Wikipedia entry for the Tor functor -- it seems to offer another interesting example of Hopf algebras appearing in the fundamental structures of something we're deeply interested in elsewhere (in this case, Homological algebra):
\begin{quote}
	For a commutative ring $R$ with a homomorphism onto a field $k$, $\text{Tor}_\bullet^R(k, k)$ is a graded-commutative Hopf algebra over $k$. If $R$ is a Noetherian local ring with residue field $k$ then the dual Hopf algebra to $Tor_\bullet^R(k,k)$ is $\text{Ext}^\bullet_R(k, k)$
\end{quote}
The reference given looks a bit specialized but it might be worth making something of this connection.

Pretty sure this block will be just one course but perhaps it will be two, an introductory one and a more advanced one with Homological Algebra as a prerequisite.

\begin{itemize}
	\item[]{Dascalescu, Nastasescu and Raianu, \textit{Hopf Algebras: An Introduction}}
\end{itemize}


\end{document}