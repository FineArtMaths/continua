\documentclass[article]{article}
\usepackage{mathtools}
\usepackage{amssymb}
\usepackage{amsthm}
\begin{document}

\title{Structure: Outline and Bibliography}
\author{Rich Cochrane}
\maketitle

\section{\textit{Hors d'oeuvre}}

Eventually there should be a course that gathers up all the absolutely essential algebra for the early continua courses, allowing a student to bypass the courses below that go into more detail. My approach is to gather up the prerequisites as we go and make this one of the last courses I write. Some things I know will need to go in there:

\begin{itemize}
	\item{Categories}
	\item{Sets}
	\item{Groups}
	\item{Exact sequences and homology groups}
	\item{Rings and fields}
	\item{Modules and vector spaces}
\end{itemize}

In general the student probably needs, for each kind of object,

\begin{itemize}
	\item{A definition of the object and its morphisms}
	\item{A few canonical examples}
	\item{What the kernels and cokernels look like}
	\item{Examples of subobjects, products, quotients}
\end{itemize}

The idea is to trade off the fact that this will be a bit dry with keeping it short and to the point, probably with a view to it being used more as a quick reference than a stand-alone course of study.

It may be worth including a bit more of the language of categories since I'm not sure where else we would do this. This part could be written earlier on.

\section{Group Theory}

This course is already written but will need to be converted to \LaTeX, although it's currently very traditional and could be expanded with more interesting material.


\section{Linear Algebra}

This course is already written but will need to be converted to \LaTeX. It's actually currently two courses; I'd like to cut some of the later applications and condense it into one.

\section{Homological Algebra}

\begin{itemize}
	\item{Stammbach, \textit{A Course in Homological Algebra} (first c170pp)}
	\item{Rotman, \textit{Introduction to Homological Algebra}}
	\item{Bjork, \textit{Rings of Differential Operators}}
\end{itemize}

Rotman is huge but it may be worth it to work through the whole thing; Stammbach offers a more direct route through the essentials. This area overlaps heavily with the topology section Munkres so it might not be a whole separate course but it would be worth seeing whether, by going deeper into the abstract structure, we can demonstrate the broad usefulness of these techniques outside topology.

Bjork is a bit of a weird inclusion here but I'd like to see whether there are applications of homological algebra there that are unusual and potentially make connections to other sections of the project.

\section{Galois Theory}

\begin{itemize}
	\item{Roman, \textit{Field Theory} (first c200 pages)}
	\item{Berhuy, \textit{An Introduction to Galois Cohomology and its Applications}}
	\item{Malle and Matzat, \textit{Inverse Galois Theory}}
	\item{Pommaret, \textit{Differential Galois Theory}}
	\item{Borceux and Janelidze, \textit{Galois Theories}}
\end{itemize}

I think the theme here is that Galois theory provides a way to understand extension problems and obstructions to solving them, which gives it a homological flavour. I'm sure all these references won't be needed but I don't know this territory.

There is some rationale to these choices, although some of this may turn out to be misplaced. Roman is introductory. Berhuy makes the link with cohomology, which is one way to connect us to the main trunk of the project; another is the connection with differential algebra made by Pommaret. 

I suspect Malle and Matzat offer one route towards `Galois theory in itself' rather than as a technique for solving problems in other fields. Borceax and Janelidze appear to give a higher-level view of `Galois theory as a whole', providing more opportunities for structural connections that span multiple topics.

\section{Lie Algebras and Representaion Theory}

\begin{itemize}
	\item{Fulton and Harris, \textit{Representation Theory: A First Course}}
	\item{Hochschild, \textit{Basic Theory of Algebraic Groups and Lie Algebras}}
	\item{Brown, \textit{Cohomology of Groups}}
	\item{Pittner, \textit{Algebraic Foundations of Non-Commutative Differential Geometry and Quantum Groups}}
\end{itemize}

This will be an \textit{algebraic} course rather than a geometric look at Lie groups but there is obvious overlap and it's not yet clear to me how much of this material belongs with differential geometry. Pittenr is included specifically for advanced stuff on Lie groups and the connection with non-commutative objects that comes up in various other places.

\section{Hopf Algebras}

\begin{itemize}
	\item{Dascalescu, \textit{Hopf Algebras: An Introduction}}
\end{itemize}

Dascalescu goes slowly and has some good examples that are relevant to us. This is probably material to scatter throughout several parts of the project so it might not be a separate course. On the other hand since it comes up in several places it might be best to have a single source to learn it from.

There are lovely diagrams for understanding these -- see the note `Cohomology of the Adjoint Hopf Algebras' for an explanation with examples.

\end{document}