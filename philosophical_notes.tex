\documentclass[article]{article}
\usepackage[T1]{fontenc} 
\usepackage[latin9]{inputenc} 
\usepackage{amsmath} 
\usepackage{amstext} 
\usepackage{amsthm} 
\usepackage{amssymb}
\usepackage{mathrsfs} 
\usepackage{hyperref}
\usepackage{babel}   

\makeatletter 

\makeatother


\providecommand{\definitionname}{Definition} 
\providecommand{\theoremname}{Theorem}
\providecommand{\Cech}{\v{C}ech }
\providecommand{\Poinacre}{Poincar\'e }
\providecommand{\Kunneth}{K{\"u}nneth }

\begin{document}
	\title{Continua Project: Notes on the Background Philosophy}
	\author{Rich Cochrane}
	\maketitle
	
\section*{Remark}

This document aims to collect the main philosophical points informing the Continua project. My intention is for these to mostly be in the background of the project, keeping them out of the way of those who hold a different position but still want to use the resources for learning. I expect there will be set-aside sections of some courses that discuss these topics, but they may end up segregated into a separate book to allow a through line to be developed.

\section{Pragmatism}

`My job has been finding definitions exactly by showing what you can deduce from them' - Giovanni Sambin, from his talk: https://www.youtube.com/watch?v=SiswEyrk2h4&t=307s. 

I take Sambin to mean that definitions don't capture something that already exists -- they're made by people for the purpose of deducing theories from them. A mathematical object has the definition it does only because it leads to a theory that people find worthwhile to investigate.

Mathematics does not need foundations -- it grows `downwards' from practical concerns, not `upwards' from some set of universal ideas. Category theory emphasises this -- it starts as a set of observations about how successful mathematicians operate in the world and develops into a toolbox for doing mathematics well. It doesn't present itself as a foundation and shouldn't be mistaken for one. 

\section{The Nature of the Continuum}

\section{The Nature of Structure}

\section{The Role of Rationality}

Logic and probability

\end{document}

