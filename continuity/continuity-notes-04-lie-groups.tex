\documentclass[oneside,english]{amsbook} 
\usepackage[T1]{fontenc} 
\usepackage[latin9]{inputenc} 
\usepackage{amsmath} 
\usepackage{amstext} 
\usepackage{amsthm} 
\usepackage{amssymb}
\usepackage{mathrsfs} 
\usepackage{hyperref}

\makeatletter 

\numberwithin{section}{chapter} 
\theoremstyle{plain} 
\newtheorem{thm}{\protect\theoremname}   
\theoremstyle{definition}   
\newtheorem{defn}[thm]{\protect\definitionname}

\makeatother

\usepackage{babel}   
\providecommand{\definitionname}{Definition} 
\providecommand{\theoremname}{Theorem}
\providecommand{\Cech}{\v{C}ech }
\providecommand{\Poinacre}{Poincar\'e }
\providecommand{\Kunneth}{K{\"u}nneth }

\begin{document}
	
	\title{Lie Groups -- Notes}
	
	\maketitle
	
	\tableofcontents
	
	\chapter*{About this Document}
	
	This is part of \href{https://github.com/FineArtMaths/continua}{the continua project}. The project is broken down into `modules', each covering a specific topic. Each module ultimately becomes one of more courses. The approach is:
	
	\begin{itemize}
		\item{Identify a topic that should be its own module (this is recorded in the `overview' documents in each main folder)}
		\item{Create a `notes' document assembling the technical material for each module in a fairly condensed form, but with some indication of how the pedagogy might go (that's what you're looking at now)}
		\item{Create one or more coursebooks that contain the technical material along with motivation, philosophical reflections, pictures, examples, intuitive explanations and so on.}
	\end{itemize}
	
	When the coursebooks are complete, the notes document is no longer needed and will probably be deleted. So the fact that you're looking at this means this is a work in progress -- it is incomplete, disorganised and probably full of errors.
	
	\part{Introduction to Lie Groups}
	
	\chapter{The Lie Functor}
	
	This is how H and G get started with the manifolds part of their book (9.1). Starting the basic definitions from the categorical construction might be a nice way to make it clear this isn't something arbitrary.
		\section{Lie Groups}
		\section{The Lie Algebra of a Lie Group}
		\section{Principal Bundles}

	\chapter{The Exponential Map}
		It's worth looking at Arnold ch 13-17 on this alongside Fecko as he has a very strong focus on calculation.
	
		\section{One-Parameter Subgroups}
		\section{The Exponent}
			From Fecko 11.4: `the motion along a one-parameter subgroup is the image with respect to exp of the uniform straight-line motion in the Lie algebra'.
		\section{The Logarithm}
		\section{Invariant Forms}
			Fecko 11.5 and 11.6 -- connection with differential forms and integration on the manifold. See also Azcarraga \& Izquirerdo, esp 1.6.

	\chapter{The Adjoint Representation}
		I don't want to get into the representation theory here (that belongs on the algebra course) but Fecko 12.3 and surrounding sections gives a nice structural feature that's worth pulling out because it relates closely to the geometry and calculus. NB Lee's \textit{Smooth Manifolds} has a section on this too.
	
	\chapter{Model Geometries}
		\section{Basic Definitions}
			This chapter and the next can take their start from Nachiketa's essay. A model geometry is a manifold equipped with a Lie group that is maximal in a specific way, and with a couple of other conditions that make it reasonably simple.
			
		\section{The Developing Map}
	
		\section{Compact Point Stabilizers}

		\section{Model Geometries in Two Dimensions}

	\chapter{Thurston's Geometrization Conjecture}

		\section{Setup}

		\section{The Eight Model Geometries}

		\section{Outline of the Proof}

\end{document}
