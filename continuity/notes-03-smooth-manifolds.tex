\documentclass[oneside,english]{amsbook} 
\usepackage[T1]{fontenc} 
\usepackage[latin9]{inputenc} 
\usepackage{amsmath} 
\usepackage{amstext} 
\usepackage{amsthm} 
\usepackage{amssymb}
\usepackage{mathrsfs} 
\usepackage{hyperref}

\makeatletter 

\numberwithin{section}{chapter} 
\theoremstyle{plain} 
\newtheorem{thm}{\protect\theoremname}   
\theoremstyle{definition}   
\newtheorem{defn}[thm]{\protect\definitionname}

\makeatother

\usepackage{babel}   
\providecommand{\definitionname}{Definition} 
\providecommand{\theoremname}{Theorem}
\providecommand{\Cech}{\v{C}ech }
\providecommand{\Poinacre}{Poincar\'e }
\providecommand{\Kunneth}{K{\"u}nneth }

\begin{document}
	
	\title{Algebraic Topology -- Notes}
	
	\maketitle
	
	\tableofcontents
	
	\chapter*{About this Document}
	
	This is part of \href{https://github.com/FineArtMaths/continua}{the continua project}. The project is broken down into `modules', each covering a specific topic. Each module ultimately becomes one of more courses. The approach is:
	
	\begin{itemize}
		\item{Identify a topic that should be its own module (this is recorded in the `overview' documents in each main folder)}
		\item{Create a `notes' document assembling the technical material for each module in a fairly condensed form, but with some indication of how the pedagogy might go (that's what you're looking at now)}
		\item{Create one or more coursebooks that contain the technical material along with motivation, philosophical reflections, pictures, examples, intuitive explanations and so on.}
	\end{itemize}
	
	When the coursebooks are complete, the notes document is no longer needed and will probably be deleted. So the fact that you're looking at this means this is a work in progress -- it is incomplete, disorganised and probably full of errors.
	
	\part{Introduction to Smooth Manifolds}
	
	\chapter{Locally Ringed Spaces}
	
	\chapter{The Tangent and Cotangent Bundles}

	\chapter{The Alternating Algebra}
	
	\chapter{Differential Forms}

	\chapter{The Exterior Derivative}

	\chapter{Integration of Forms}

	\chapter{De Rham's Theorem}
	
	\chapter{Vector Flows and Integral Curves}

	\chapter{The Lie Derivative}

	\chapter{The Frobenius Theorem}

\end{document}
