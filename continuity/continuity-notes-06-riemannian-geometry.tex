\documentclass[oneside,english]{amsbook} 
\usepackage[T1]{fontenc} 
\usepackage[latin9]{inputenc} 
\usepackage{amsmath} 
\usepackage{amstext} 
\usepackage{amsthm} 
\usepackage{amssymb}
\usepackage{mathrsfs} 
\usepackage{hyperref}

\makeatletter 

\numberwithin{section}{chapter} 
\theoremstyle{plain} 
\newtheorem{thm}{\protect\theoremname}   
\theoremstyle{definition}   
\newtheorem{defn}[thm]{\protect\definitionname}

\makeatother

\usepackage{babel}   
\providecommand{\definitionname}{Definition} 
\providecommand{\theoremname}{Theorem}
\providecommand{\Cech}{\v{C}ech }
\providecommand{\Poinacre}{Poincar\'e }
\providecommand{\Kunneth}{K{\"u}nneth }

\begin{document}
	
	\title{Riemannian Geometry -- Notes}
	
	\maketitle
	
	\tableofcontents
	
	\chapter*{About this Document}
	
	This is part of \href{https://github.com/FineArtMaths/continua}{the continua project}. The project is broken down into `modules', each covering a specific topic. Each module ultimately becomes one of more courses. The approach is:
	
	\begin{itemize}
		\item{Identify a topic that should be its own module (this is recorded in the `overview' documents in each main folder)}
		\item{Create a `notes' document assembling the technical material for each module in a fairly condensed form, but with some indication of how the pedagogy might go (that's what you're looking at now)}
		\item{Create one or more coursebooks that contain the technical material along with motivation, philosophical reflections, pictures, examples, intuitive explanations and so on.}
	\end{itemize}
	
	When the coursebooks are complete, the notes document is no longer needed and will probably be deleted. So the fact that you're looking at this means this is a work in progress -- it is incomplete, disorganised and probably full of errors.
	
	\part{Introduction to Riemannian Geometry}
	
	\chapter{Introduction}
		\section{Geometric Problems in Flat Space}

			It seems like it might be nice to start this course back in flat space, using the metric we have there `for free'. One of the most visible problems in flat space is the absence of a canonical isomorphism between the tangent and cotangent spaces. So we can start with the Hodge star, then perhaps codifferentials and  the Laplace-deRham Operator that are described in Abraham and Marsden, \emph{Foundations of Mechanics} (p.153). 
			
			Perhaps even some elements of Riemannian geometry that are easy to develop. This should motivate the coming discussion in curved space. This follows the same pattern as Smooth Manifolds, or the Mod 2 part of the Intro Algebraic Topology course.

	\chapter{Principal Bundles and Connections}
		
		See Sontz ch 9 and 10. Ramanan covers it in a completely different way in his ch 5, 6 and 7. 
		
		Banchoff and Lovett ch 6 is very concrete about the metric but this is in ambient flat space -- Lovett ch 5 does the equivalent for manifolds.
	
	\chapter{Curvature}
	
		Sontz does this in ch 11.
		
		Lee has a whole chapter on the Gauss–Bonnet Theorem

	\chapter{Applications to Optimization Problems}
	
		Geodesics, minimal surfaces
	
	\chapter{Sympleptic and K\:ahler Manifolds}
		Nicolaescu ch 3 has a section on this that might be nice to cover here as a connection with applications.
		

	
\end{document}
