\documentclass[oneside,english]{amsbook} 
\usepackage[T1]{fontenc} 
\usepackage[latin9]{inputenc} 
\usepackage{amsmath} 
\usepackage{amstext} 
\usepackage{amsthm} 
\usepackage{amssymb}
\usepackage{mathrsfs} 
\usepackage{hyperref}

\makeatletter 

\numberwithin{section}{chapter} 
\theoremstyle{plain} 
\newtheorem{thm}{\protect\theoremname}   
\theoremstyle{definition}   
\newtheorem{defn}[thm]{\protect\definitionname}

\makeatother

\usepackage{babel}   
\providecommand{\definitionname}{Definition} 
\providecommand{\theoremname}{Theorem}
\providecommand{\Cech}{\v{C}ech }
\providecommand{\Poinacre}{Poincar\'e }
\providecommand{\Kunneth}{K{\"u}nneth }

\begin{document}
	
	\title{Smooth Manifolds -- Notes}
	
	\maketitle
	
	\tableofcontents
	
	\chapter*{About this Document}
	
	This is part of \href{https://github.com/FineArtMaths/continua}{the continua project}. The project is broken down into `modules', each covering a specific topic. Each module ultimately becomes one of more courses. The approach is:
	
	\begin{itemize}
		\item{Identify a topic that should be its own module (this is recorded in the `overview' documents in each main folder)}
		\item{Create a `notes' document assembling the technical material for each module in a fairly condensed form, but with some indication of how the pedagogy might go (that's what you're looking at now)}
		\item{Create one or more coursebooks that contain the technical material along with motivation, philosophical reflections, pictures, examples, intuitive explanations and so on.}
	\end{itemize}
	
	When the coursebooks are complete, the notes document is no longer needed and will probably be deleted. So the fact that you're looking at this means this is a work in progress -- it is incomplete, disorganised and probably full of errors.
	
	\part{Introduction to Smooth Manifolds}
	
	\chapter*{General Approach}
	
	The idea is to develop as much as we can in flat space first,, with concrete examples and calculations. This will include introducing the idea of a limit, which I'd like to approach in the SDG manner without being dogmatic.
	
	In the same spirit, I'd like to introduce manifolds as locally ringed spaces and do things with that setup (including developing a little bit of sheaf cohomology) but keep most of the material quite traditional. 

	\chapter{Calculus in Flat Space}
	
		\section{Linear Algebra Review}
			\subsection{Dual Space and Inner Product}
			\subsection{Covariance and Contravariance}
			Recall scalar fields and curves
			Include algebra of scalar fields. Note there is no algebra of curves. 
			\subsection{The Exterior Algebra}
		\subsection{Fibre Bundles}
		Scalar Fields, vector fields and covector fields as sections of bundles
		
		\section{Locally Ringed Spaces}
		
			\begin{defn} 
				A \textbf{sheaf of rings} $\mathscr{O}(X)$ on a topological space $X$ assigns a ring to each open set of $X$ such that the sheaf axioms hold, that is (here $U$, $V$ and $W$ are open sets in $X$):
				\begin{itemize}
					\item For every pair $U\subseteq V$ there is a ring homomorphism $res_U^V: \mathscr{O}(V)\to \mathscr{O}(U)$
					\item For each open set $U$, $res_U^U$ is the identity map
					\item $res_U^V \circ res_V^W = res_U^W$ (a sort of `triangle inequality')
				\end{itemize}
			\end{defn}
			
			\begin{defn} 
				A \textbf{ringed space} is a topological space $X$ equipped with a sheaf of rings $\mathscr{O}(X)$, which is called its \textbf{structure sheaf}
			\end{defn}
			
			\begin{defn} 
				A \textbf{local ring} is a ring that has a unique maximal ideal.
			\end{defn}
			
			\begin{defn} 
				A \textbf{locally ringed space} is a ringed space such that the stalk of the structure sheaf at every point is a local ring.
			\end{defn}

		\section{Infinitesimals and Limits}
		
		The first part of Bell's essay here is a good approach: \url{https://citeseerx.ist.psu.edu/document?repid=rep1&type=pdf&doi=9100113e2c74c99c5262c6a686a63b166dc9aaa7}
		
		\section{Differential Forms}
		
		\url{https://rigtriv.wordpress.com/2008/04/14/differential-forms-and-the-canonical-bundle/}
	
		\section{The Exterior Derivative}
	
		\section{Integration of Differential Forms}

	\chapter{Differential Calculus on Manifolds}
	
		\section{Smooth Manifolds}

		\section{Diffeomorphisms}

		\section{Closed and Exact Forms}
		\section{De Rham Cohomology}
		\section{The Boundary Operator}

	\chapter{Integral Calculus on Manifolds}

		\section{Pushforwards and Pullbacks}
		\section{Integration of Forms}
		\section{Stokes's Theorem}

	\chapter{The Three-Dimensional Case}
	
		A brief chapter where we show that normal vector calculus (as far as we can express it without a metric!) just `drops out of' all the work we've already done. Should be very concrete and rather easy, with plenty of opportunities for reinforcement.

	\chapter{Vector Flows and Integral Curves}
	
		\section{Differential Equations}

		\section{Foliations}

		\section{The Lie Derivative}
	
		\section{The Frobenius Theorem}



\end{document}
